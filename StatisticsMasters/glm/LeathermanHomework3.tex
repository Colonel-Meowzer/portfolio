% Created 2020-02-03 Mon 20:08
% Intended LaTeX compiler: pdflatex
\documentclass[11pt]{article}
\usepackage[utf8]{inputenc}
\usepackage[T1]{fontenc}
\usepackage{graphicx}
\usepackage{grffile}
\usepackage{longtable}
\usepackage{wrapfig}
\usepackage{rotating}
\usepackage[normalem]{ulem}
\usepackage{amsmath}
\usepackage{textcomp}
\usepackage{amssymb}
\usepackage{capt-of}
\usepackage{hyperref}
\author{Dustin Leatherman}
\date{\today}
\title{Homework \#3}
\hypersetup{
 pdfauthor={Dustin Leatherman},
 pdftitle={Homework \#3},
 pdfkeywords={},
 pdfsubject={},
 pdfcreator={Emacs 26.3 (Org mode 9.4)}, 
 pdflang={English}}
\begin{document}

\maketitle
\tableofcontents



\section{1}
\label{sec:org5a031bc}
\subsection{a}
\label{sec:orgb4404c1}
\begin{quote}
Estimate the dispersion parameter.
\end{quote}
For Binomial or Poisson distributions, the dispersion parameter can be estimated
by the \emph{Pearson Chi-Squared Statistic}

\begin{equation}
\begin{split}
\hat \phi = \frac{X^2}{n - p} = 1.2927
\end{split}
\end{equation}

\subsection{b}
\label{sec:org4d8a40a}
\begin{quote}
Compute Full Log Likelihood and BIC for the model (hint AIC is given)
\end{quote}

\begin{equation}
\begin{split}
AIC = & -2 l(\hat \pi; y) + 2p\\
BIC = & -2 l(\hat \pi; y) + p \times log(n)
\end{split}
\end{equation}

\textbf{Log Likelihood}

\begin{equation}
\begin{split}
92.2094 = & -2 \ l(\hat \pi ; y) + 2p\\
46.1047 = &  - l(\hat \pi; y) + 2\\
-44.1047 = & l(\hat \pi; y)
\end{split}
\end{equation}

\textbf{BIC}

\begin{equation}
\begin{split}
-44.1047 \times -2 + 2 \times log(22) = 94.39148
\end{split}
\end{equation}
\section{2}
\label{sec:orgb410f77}
\begin{quote}
Let Y = number of ACFs in the rat colons and x = sacrificed times (endtime, 6,
12, and 18). Compute the predicted probabilities for Y = 2, 4, 8, and x = 12.
\end{quote}


\begin{equation}
\begin{split}
\lambda = & exp(-0.3215 + 0.1192 \times 12) = 3.031022\\
P(y) = & \frac{e^{- \lambda} \lambda^y}{y!}\\
= & \frac{e^{- 3.031022} 3.031022^y}{y!}\\
\end{split}
\end{equation}

\begin{center}
\begin{tabular}{rrr}
 & Y & P(y/ x = 12)\\
\hline
1 & 2 & 0.2217134\\
2 & 4 & 0.1697419\\
3 & 8 & 0.0085278\\
\end{tabular}
\end{center}
\subsection{a}
\label{sec:org99d1a83}
\begin{quote}
How do we interpret \(\hat \beta_1 = 0.1192\)?
\end{quote}

\begin{equation}
\begin{split}
exp(0.1192) = 1.126595
\end{split}
\end{equation}

The incident rate ratio increases by 120\% for each one-unit increase in
sacrifice time.

\section{3}
\label{sec:org1b9402a}
\begin{quote}
To study factors that affect the recurrence of heart attacks (HA), an
investigator collected data from 20 HA victims. The investigator fit a logistic regression model with an indicator of a second
HA within one year (1 = HA; 0 = no HA) as the binary outcome. There are two
predictors: \(x_1 = 1\) if the patient completed an anger management program; 0
else. \(x_2 =\) anxiety score (0 = low, 100 = high). Computer output is given
below:

\begin{center}
\begin{tabular}{lrrrr}
 & Estimate & Std Err & Z value & P Value\\
\hline
Intercept & -6.36 & 3.21 & -1.98 & 0.05\\
X1 & -1.02 & 1.17 & -0.88 & 0.38\\
X2 & 0.12 & 0.06 & 2.17 & 0.03\\
\end{tabular}
\end{center}
\end{quote}

\subsection{a}
\label{sec:orgc87dcff}
\begin{quote}
In terms of \(x_1\) and \(x_2\), what are the odds of a patient having a second
heart attack?
\end{quote}

\begin{equation}
\begin{split}
\omega_{AB} = & \frac{\omega_A}{\omega_B}\\
= & \frac{e^{X0 + X1 \times 1 + X2 \times A}}{e^{X0 + X1 \times 0 + X2 \times B}}\\
= & e^{X1 (1 - 0) + X2(A = B)}\\
= & e^{X1 + X2 (A - B)}
\end{split}
\end{equation}

\subsection{b}
\label{sec:org43fb698}
\begin{quote}
What is the probability of a second heart attack for a patient that has
completed an anger management program and scored a 100 on the anxiety test?
\end{quote}

\begin{equation}
\begin{split}
\pi = & \frac{e^\eta}{1 + e^\eta}\\
= & \frac{e^{-6.36 - 1.02 \times 1 + 0.12 \times 100}}{1 + e^{-6.36 - 1.02 \times 1 + 0.12 \times 100}}\\
= & 0.9902433
\end{split}
\end{equation}


\subsection{c}
\label{sec:orgfd5ff17}
\begin{quote}
For patients that have completed the anger management program, is high anxiety
associated with an increased probability of a second heart attack?
\end{quote}

Regardless of whether or not a patient has completed the anger management
program, there is moderate evidence that a higher anxiety score is associated with an
increased risk of a second heart attack (p-value = 0.03).

\subsection{d}
\label{sec:orgd3950f7}
\begin{quote}
Is there statistical evidence that an anger management program is associated
with a reduction in the probability of a second heart attack? Explain.
\end{quote}

There is no evidence that completion of the anger management program is
associated with reduced probability of a second heart attack (p-value = 0.38).
The confidence interval for the anger management predictor encompasses 0
indicating that there is no conclusive effect on the estimated predicted probability.

\subsection{e}
\label{sec:org6bae1fd}
\begin{quote}
Explain why linear regression is in appropriate for modeling the probability of
a second heart attack.
\end{quote}

Linear Regression may yield invalid values, in this case, an estimated
probability using Linear Regression may be outside the interval of \([0,1]\).
Linear Regression could be used to estimate a relative score or value that could
be interpreted in a similar fashion but there would be no guarantee on the range
of scores that would occur; whereas when modeling probabilities, the probability
is guaranteed to be between 0 and 1.

\section{4}
\label{sec:org27b0f3b}
\begin{quote}
Let Y be a binomial distribution. Show taht Y has the exponential distribution
of the form:

$$
f(y; \theta) = s(y) y(\theta) exp(a(y) b(\theta));
$$

this can be rewritten

$$
f(y; \theta) = exp(a(y) (\theta) + c(\theta) + d(y))
$$
\end{quote}


\subsection{a}
\label{sec:org5fb5e36}
\begin{quote}
Clearly identify the link function, \(b(\theta)\)
\end{quote}

\begin{equation}
\begin{split}
f(y; \pi) = & {n \choose y} \pi^y (1 - \pi)^{n - y}\\
= & exp(y \ log(\pi) + (n - y) log(1 - \pi) + log({n \choose y}))\\
= & exp(y \ log (\pi) + n \ log(1 - \pi) - y \ log(1 - \pi) + log({n \choose y}))\\
= & exp(y \ (log(\pi) - log(1 - \pi)) + n \ log (1 - \pi) + log({n \choose y}))\\
= & exp(y \ log(\frac{\pi}{1 - \pi}) + n \ log (1 - \pi) + log({n \choose y}))\\
\end{split}
\end{equation}

\(a(y) = y\)

\(b(\theta) = log(\frac{\pi}{1 - \pi})\)

\(c(\theta) = n log(1 - \pi)\)

\(d(y) = log({n \choose y})\)

\section{5}
\label{sec:org1bee55a}
\begin{quote}
For games in baseball's National League during nine decades: The following table
shows the percentage of times that the starting pitcher pitched a complete game.

\begin{center}
\begin{tabular}{rrr}
 & Decade\textsubscript{complete} & Percent\\
\hline
1 & 1900-1909 & 72.7\\
2 & 1910-1919 & 63.4\\
3 & 1920-1939 & 50\\
4 & 1930-1939 & 44.3\\
5 & 1940-1949 & 41.6\\
6 & 1950-1959 & 32.8\\
7 & 1960-1969 & 27.2\\
8 & 1970-1979 & 22.5\\
9 & 1980-1989 & 13.3\\
\end{tabular}
\end{center}
\end{quote}

\subsection{a}
\label{sec:orge8dbf12}
\begin{quote}
Treating the number of games as the same in each decade, the ML fit of the
linear probability model is \(\hat p = 0.7578 - 0.0694 x\), where x = decade
[1:9]. Interpret
\end{quote}

Each additional decade starting at 1900 is associated with a 6.94\% \emph{decrease} in
the percentage of times that the starting pitcher pitched a complete game.

\subsection{b}
\label{sec:org7ddbf46}
\begin{quote}
Substituting \(x = 10,11,12\), predict the percentage of complete games for the
next three decades. Are these predictions plausible? Why?
\end{quote}

\begin{equation}
\begin{split}
0.7578 - 0.0694 \times 11 = & -0.0056\\
0.7578 - 0.0694 \times 12 = & -0.075\\
0.7578 - 0.0694 \times 13 = & -0.1444\\
\end{split}
\end{equation}

These predictions are not plausible because they fall outside the range between
0 and 1. This is one of the reasons why linear regression is not suitable for
predicting probabilities.

\subsection{c}
\label{sec:org0bc19f6}
\begin{quote}
The ML Fit with logistic regression is

$$
\hat p = exp(1.148 - 0.315 x) /(1 + exp(1.148 - 0.315 x))
$$

Obtain for \(x = 10,11,12\). Are these more plausible?
\end{quote}

\begin{equation}
\begin{split}
exp(1.148 - 0.315 \times 10) /(1 + exp(1.148 - 0.315 \times 10)) = & \ 0.1189931\\
exp(1.148 - 0.315 \times 11) /(1 + exp(1.148 - 0.315 \times 11)) = & \ 0.08972478\\
exp(1.148 - 0.315 \times 12) /(1 + exp(1.148 - 0.315 \times 12)) = & \ 0.06710713\\
\end{split}
\end{equation}

These are more plausible since the values are valid (between 0 and 1) and still show a
decreasing probability over time.

\section{6}
\label{sec:org6a9d4b7}
\begin{quote}
Show that the following probability density functions belong to the exponential family:
\end{quote}

\subsection{a}
\label{sec:orgc756119}
\begin{quote}
Pareto distribution

$$
f(y: \theta) = \thetaY^{-\theta-1}
$$
\end{quote}

\begin{equation}
\begin{split}
f(y; \theta) = & \ \theta Y^{-\theta - 1}\\
= & \ exp((- \theta - 1) \ log(y) + log(\theta))\\
= & \ exp(- \theta log(y) - log(y) + log(\theta))
\end{split}
\end{equation}


\begin{equation}
\begin{split}
a(y) = & \ - log(y)\\
b(\theta) = & \ \theta\\
c(\theta) = & \ log(\theta)\\
d(y) = & \ - log(y)
\end{split}
\end{equation}

\subsection{b}
\label{sec:orgdf58f98}
\begin{quote}
Exponential distribution

$$
f(y; \theta) = \theta \ exp(-y \theta)
$$
\end{quote}

\begin{equation}
\begin{split}
f(y; \theta) = & \ \theta \ exp(-y \theta)\\
= & \ exp(log(\theta) - y \theta)
\end{split}
\end{equation}


\begin{equation}
\begin{split}
a(y) = & \ -y\\
b(\theta) = & \ \theta\\
c(\theta) = & \ log(\theta)\\
d(y) = & \ 0
\end{split}
\end{equation}

\section{7}
\label{sec:orgb8ce061}
\begin{quote}
The following associations can be described by generalized linear models. For
each one:
\begin{enumerate}
\item Identify the response variable and the explanatory variables
\item Select a probability distribution for the response (justifying your choice)
\item Write down the linear component
\end{enumerate}
\end{quote}

\subsection{a}
\label{sec:org6a3e66c}
\begin{quote}
The effect of age, sex, height, mean daily food intake, and mean daily energy
expenditure on a person's weight.
\end{quote}

\begin{enumerate}
\item A person's weight.
\item t-distribution since weight is a nominal value with no inherent limitations
in terms of range of values.
\item \(\hat{weight} = \beta_0 + \beta_1 \ age + \beta_2 \ isMale + \beta_3 \ height + \beta_4 \ avgDailyFoodIntake + \beta_5 \ avgDailyEnergyExpend\)
\end{enumerate}

\subsection{b}
\label{sec:orgf35bdc0}
\begin{quote}
The proportion of laboratory mice that become infected after exposure to
bacteria when five different exposure levels are used and 20 mice are exposed at
each level.
\end{quote}

\begin{enumerate}
\item Proportion of infected laboratory mice
\item Binomial Distribution since a mouse can either be infected or not infected.
\item \(\hat{infected} = \beta_0 + \beta_1 \ exp1 + \beta_2 \ exp2 + \beta_3 \
   exp3 + \beta_4 \ exp4 + \beta_5 \ exp5\) where exp1 through exp5 are indicator variables (1 when exposed at a given level; 0 otherwise).
\end{enumerate}

\subsection{c}
\label{sec:orgda0b6be}
\begin{quote}
The association between the number of trips per week to the supermarket for a
household and the number of people in the household, the household income, and
the distance of the supermarket.
\end{quote}

\begin{enumerate}
\item The number of trips per week to the supermarket.
\item Poisson or Negative Binomial Distribution. If a Poisson model is fit and
there is over-dispersion, then a Negative Binomial Distribution may be a
better fit. Both Poisson and Negative Binomial are useful distributions for
modeling \emph{count} data, which this response variable is.
\item \(\hat{tripsPerWeek} = \beta_0 + \beta_1 \ numPeople + \beta_2 \ income + \beta_3 \ distance\)
\end{enumerate}
\end{document}
