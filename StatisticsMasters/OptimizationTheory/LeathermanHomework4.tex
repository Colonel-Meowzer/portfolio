% Created 2020-04-26 Sun 15:10
% Intended LaTeX compiler: pdflatex
\documentclass[11pt]{article}
\usepackage[utf8]{inputenc}
\usepackage[T1]{fontenc}
\usepackage{graphicx}
\usepackage{grffile}
\usepackage{longtable}
\usepackage{wrapfig}
\usepackage{rotating}
\usepackage[normalem]{ulem}
\usepackage{amsmath}
\usepackage{textcomp}
\usepackage{amssymb}
\usepackage{capt-of}
\usepackage{hyperref}
\author{Dustin Leatherman}
\date{\today}
\title{Homework 4}
\hypersetup{
 pdfauthor={Dustin Leatherman},
 pdftitle={Homework 4},
 pdfkeywords={},
 pdfsubject={},
 pdfcreator={Emacs 26.3 (Org mode 9.4)}, 
 pdflang={English}}
\begin{document}

\maketitle
\tableofcontents

\section{Part 1}
\label{sec:orgcd379cf}
\begin{quote}
Make a Table. The first column of this table should contain values of m, the
second column should be \(\epsilon_{median}\). Note that the median of \(\epsilon\)
depends on m.

Change the number of iterations to 20. For each value of m, (where \(A \in R^{m
\times 4m}\)), run the program 5 times and record the median of \(\epsilon\). Do
this for \(m = 200, 190, 180, 170, 160, 150, 140, 130, 106, 105, 104, 103\)
\end{quote}

\begin{center}
\begin{tabular}{rr}
m & \(\epsilon_{median}\)\\
\hline
200 & 1.90357e-09\\
190 & 4.83011e-09\\
180 & 5.91492e-08\\
170 & 3.16302e-08\\
160 & 4.70064e-08\\
150 & 1.56469e-06\\
140 & 1.07261e-07\\
130 & 6.72136e-08\\
106 & 371.846\\
105 & 251.155\\
104 & 477.818\\
103 & 998.845\\
\end{tabular}
\end{center}


\begin{verbatim}
%%% Iterative Hard Threshold algorithm %%%

mrows = [200, 190, 180, 170, 160, 150, 140, 130, 106, 105, 104, 103]
sparse = 9;
T = 20;

for m = mrows

  rows = m;
  N = 4 * rows;
  E = [];
  for i = 1:5
    Amat = gen_rand_mtx(rows, N);
    x_true = gen_x_true(N);

    yvect = Amat * x_true;    %%% y = Ax
    xvect = zeros(N, 1);

    for k = 1:T
      xvect = est_x(xvect, yvect, Amat, N, sparse);
    end

    is_conditioned(Amat);

    % [x_true([1:50],1)   xvect([1:50],1)];

    error = get_err(x_true, xvect);
    E = [E; error];
  end

  fprintf("Dimension: %i\n", m)
  fprintf("Median: %i\n", median(E))
end
\end{verbatim}

\section{Part 2}
\label{sec:orgb0db5c3}

\begin{quote}
Explore IHT with MatLab. Try changing the number of rows or columns of the
matrix A. Try changing the 9-sparse vector to 15-spares vector. For example, can
IHT reconstruct a 20-sparse vector with \(A \in R^{200 \times 500}\)? That is just
one example, you can think of many more interesting questions.

Describe carefully at least 2 interesting scenarios that you have tried. What
did you observe?
\end{quote}

\subsection{Estimate Row Number}
\label{sec:org899ffba}

Davies describes a lower bound for \(M \geq c s log(N / s)\) where c is some
constant. When \(M\) meets this criteria, then ``a random construction of \(A\) can
achieve the RIP required\ldots{}with high probability''(Davies, p6).

\url{https://arxiv.org/pdf/0805.0510.pdf}

Trying this with various \(N = [300, 500, 1000, 2000, 4000, 8000, 16000]\) and
various sparsity levels showed acceptable low median error amounts; however, only
a 16-sparse is shown.

For s = 16,

\begin{center}
\begin{tabular}{rrr}
N & Estimate M & Median Error\\
\hline
300 & 282 & 5.93634e-06\\
500 & 331 & 2.76957e-06\\
1000 & 397 & 5.85607e-09\\
2000 & 464 & 6.79957e-09\\
8000 & 597 & 3.85131e-10\\
16000 & 664 & 2.10875e-08\\
\end{tabular}
\end{center}


\begin{verbatim}

%%% Iterative Hard Threshold algorithm %%%
sparse = 16;
T = 50;
N_LIST = [300, 500, 1000, 2000, 8000, 16000];
% c gt 5. [Davies, Blumensath]
c = 6;
for N = N_LIST
    % based on Davies, Blumensath
    m = ceil(c * sparse * log(N / sparse));
    E = [];
    for i = 1:5
        Amat = gen_rand_mtx(m, N);
        x_true = gen_x_true(N);

        yvect = Amat * x_true;    %%% y = Ax
        xvect = zeros(N, 1);

        for k = 1:T
            xvect = est_x(xvect, yvect, Amat, N, sparse);
        end

        is_conditioned(Amat);

        % [x_true([1:50],1)   xvect([1:50],1)];

        error = get_err(x_true, xvect);
        E = [E; error];
    end

    fprintf("Dimension: %i\n", m)
    fprintf("Median: %i\n", median(E))
end
\end{verbatim}

\subsection{Estimate Iteration Number}
\label{sec:org4aac82e}

Let \(y^s\) be the optimal s-sparse vector for y. The estimated error can be
described by:

$$
\epsilon_s = \|y - y^s\|_2 + \frac{1}{\sqrt{s}} \|y - y^s\|_1 + \|e\|_2
$$

and the estimated iterations by

$$
k^* = log_2(\frac{\|y^s\|_2}{\epsilon_s})
$$

This formula gives negative numbers since \(\epsilon_s\) is large, meaning that
there is a lot of noise in the vector that \(y^s\) is not explaining. The absolute
value of k is used instead since it is positive.

For s = 16,

\begin{center}
\begin{tabular}{rrr}
N & Estimate M & Median Error\\
\hline
300 & 282 & 0.217618\\
500 & 331 & 0.198818\\
1000 & 397 & 0.129009\\
2000 & 464 & 0.113197\\
8000 & 597 & 0.089262\\
16000 & 664 & 0.0779062\\
\end{tabular}
\end{center}

The median error is greater than the values provided above but the amount of
iterations and thus compute have significantly reduced with the median error
still being in an acceptable range.

\begin{quote}
Note: I do not understand why estimated algorithm for iterations ``works''. It is
only producing negative values since the noise in the y vector is large.
\end{quote}

\begin{verbatim}
%%% Iterative Hard Threshold algorithm %%%

sparse = 16;
N_LIST = [300, 500, 1000, 2000, 8000, 16000];
% c gt 5. [Davies, Blumensath]
c = 6;
for N = N_LIST
    % based on Davies, Blumensath
    m = ceil(c * sparse * log(N / sparse));
    E = [];
    for i = 1:5
        Amat = gen_rand_mtx(m, N);
        x_true = gen_x_true(N);

        yvect = Amat * x_true;    %%% y = Ax
        xvect = zeros(N, 1);

        % signal-to-noise ratio for sparse estimates
        % this value is almost always negative so I am not sure why this formula should work.
        est_iter = log2(norm(s_approx(yvect, sparse, m)) / est_err(yvect, sparse, m));
        fprintf('Estimated Iterations: %i\n', est_iter)
        % est_iter is usually a negative float so taking abs() and rounding up.
        for k = 1:ceil(abs(est_iter))
            xvect = est_x(xvect, yvect, Amat, N, sparse);
        end

        is_conditioned(Amat);

        % [x_true([1:50],1)   xvect([1:50],1)];

        error = get_err(x_true, xvect);
        E = [E; error];
    end

    fprintf('Dimension: %i\n', m)
    fprintf('Median: %i\n', median(E))
end
% Estimated Error [Davies, Blumensath]
% Assumes e is a unit vector. i.e. ||e||_2 = 1
% Large values indicate lots of noise.
function err = est_err(yvect, sparse, m)
   noise = yvect - s_approx(yvect, sparse, m);
   err = norm(noise) + (norm(noise, 1) / sqrt(sparse)) + 1;
end
\end{verbatim}
\section{Original Code}
\label{sec:org1004933}
\begin{verbatim}
%%% Iterative Hard Threshold algorithm %%%

sparse = 9;
rows = 200;
N = 4 * rows;
T = 50;
%%%%%%%%%%%%%%%%%%%%%%%%%%%%%%%%%%%%%%%%%%%%%%%%
%%%%   Make the random matrix
%%%%%%%%%%%%%%%%%%%%%%%%%%%%%%%%%%%%%%%%%%%%%%%%

Amat = randn(rows,N);
Amat = 1 / sqrt(rows) * Amat;

%%%%%%%%%%%%%%%%%%%%%%%%%%%%%%%%%%%%%%%%%%%%%%%%
%%%
%%%    Make the true x vector
%%%
%%%%%%%%%%%%%%%%%%%%%%%%%%%%%%%%%%%%%%%%%%%%%%%%

x_true = zeros(N,1);
x_true(6) = 1.2;
x_true(7) = 0.7;
x_true(8) = 1.2;
x_true(15) = 0.7;
x_true(16) = -1.2;
x_true(17) = -0.7;
x_true(18) = -1.3;
x_true(19) = 1.3;
x_true(39) = 1.3;

yvect = Amat * x_true;    %%% y = Ax
xvect = zeros(N,1);

%%%      Here is how to use the sort function
%%% c = [1 3 5 7 9 2 4 6 8 10];
%%% [b, place] = sort(c, 'descend');
%%% c( place(1:4));

for k = 1:T
    uvect = xvect + Amat' * (yvect - Amat * xvect);
    [u_sort, place] = sort(abs(uvect), 'descend');

    xvect = zeros(N,1);

    % populates the xvect in descending order
    for j = 1:sparse
        xvect(place(j)) = uvect(place(j));
    end
end

%%%%%%%%%%%%%%%%%%%%%%%%%%%%%%%%%%%%%%%%%%%%%%%%%%%%%%%%%%%%%%%%%%%%%
%%%
%%% The following 3 lines check that matrix Amat is well-conditioned
%%%
%%% Think of e_max and e_min as (1+ delta_s) and (1 - delta_s) in RIP
%%%%%%%%%%%%%%%%%%%%%%%%%%%%%%%%%%%%%%%%%%%%%%%%%%%%%%%%%%%%%%%%%%%%%%

A_submat = Amat(:, [39,15,16,17,18,19,6,7,8]);

eigenvalues = eig(A_submat' * A_submat);

e_max = max(eigenvalues);
e_min = min(eigenvalues);
[e_max, e_min]

%%% Compare the true values with the predicted values.
%%% The predicted values are in the vector xvect.
%%% Show the first 50 values of the true values and predicted values.

[x_true([1:50],1)   xvect([1:50],1)]

norm(x_true - xvect)/norm(x_true)
\end{verbatim}
\section{Refactored Code}
\label{sec:org668ba45}
\begin{verbatim}
%%% Iterative Hard Threshold algorithm %%%


%%%%%%%%%%%%%%%%%%%%%%%%%%%%%%%%%%%%%%%%%%%%%%%%
%%%%   Make the random matrix
%%%%%%%%%%%%%%%%%%%%%%%%%%%%%%%%%%%%%%%%%%%%%%%%

function Amat = gen_rand_mtx(rows, N)
  Amat = randn(rows,N);
  Amat = 1 / sqrt(rows) * Amat;
end

%%%%%%%%%%%%%%%%%%%%%%%%%%%%%%%%%%%%%%%%%%%%%%%%
%%%
%%%    Make the true x vector
%%%
%%%%%%%%%%%%%%%%%%%%%%%%%%%%%%%%%%%%%%%%%%%%%%%%

function x_true = gen_x_true(N)
  x_true = zeros(N,1);
  x_true(6) = 1.2;
  x_true(7) = 0.7;
  x_true(8) = 1.2;
  x_true(15) = 0.7;
  x_true(16) = -1.2;
  x_true(17) = -0.7;
  x_true(18) = -1.3;
  x_true(19) = 1.3;
  x_true(39) = 1.3;
end


%%%      Here is how to use the sort function
%%% c = [1 3 5 7 9 2 4 6 8 10];
%%% [b, place] = sort(c, 'descend');
%%% c( place(1:4));

function xvect = est_x(xvect, yvect, Amat, N, sparse)
  uvect = xvect + Amat' * (yvect - Amat * xvect);

  xvect = s_approx(uvect, sparse, N);
end

function v_sparse = s_approx(v, s, N)
  v_sparse = zeros(N, 1);
  [v_sort, place] = sort(abs(v), 'descend');
  % populates v_sparse in descending order
  for j = 1:s
      v_sparse(place(j)) = v(place(j));
  end
end

%%%%%%%%%%%%%%%%%%%%%%%%%%%%%%%%%%%%%%%%%%%%%%%%%%%%%%%%%%%%%%%%%%%%%
%%%
%%% The following 3 lines check that matrix Amat is well-conditioned
%%%
%%% Think of e_max and e_min as (1+ delta_s) and (1 - delta_s) in RIP
%%%%%%%%%%%%%%%%%%%%%%%%%%%%%%%%%%%%%%%%%%%%%%%%%%%%%%%%%%%%%%%%%%%%%%

function [e_max, e_min] = is_conditioned(Amat)
  A_submat = Amat(:, [39,15,16,17,18,19,6,7,8]);

  eigenvalues = eig(A_submat' * A_submat);

  e_max = max(eigenvalues);
  e_min = min(eigenvalues);
end


%%% Compare the true values with the predicted values.
%%% The predicted values are in the vector xvect.
%%% Show the first 50 values of the true values and predicted values.


function error = get_err(x_true, xvect)
  error = norm(x_true - xvect)/norm(x_true);
end

%% Put it all together

sparse = 9;
rows = 100;
N = 4 * rows;
T = 50;

Amat = gen_rand_mtx(rows, N);
x_true = gen_x_true(N);

yvect = Amat * x_true;    %%% y = Ax
xvect = zeros(N, 1);

for k = 1:T
  xvect = est_x(xvect, yvect, Amat, N, sparse);
end

is_conditioned(Amat)

[x_true([1:50],1)   xvect([1:50],1)]

get_err(x_true, xvect)
\end{verbatim}
\end{document}
