% Created 2020-04-06 Mon 21:12
% Intended LaTeX compiler: pdflatex
\documentclass[11pt]{article}
\usepackage[utf8]{inputenc}
\usepackage[T1]{fontenc}
\usepackage{graphicx}
\usepackage{grffile}
\usepackage{longtable}
\usepackage{wrapfig}
\usepackage{rotating}
\usepackage[normalem]{ulem}
\usepackage{amsmath}
\usepackage{textcomp}
\usepackage{amssymb}
\usepackage{capt-of}
\usepackage{hyperref}
\author{Dustin Leatherman}
\date{\today}
\title{Optimization - Homework 1}
\hypersetup{
 pdfauthor={Dustin Leatherman},
 pdftitle={Optimization - Homework 1},
 pdfkeywords={},
 pdfsubject={},
 pdfcreator={Emacs 26.3 (Org mode 9.4)}, 
 pdflang={English}}
\begin{document}

\maketitle
\tableofcontents


\section{Problem Statement}
\label{sec:orgb484466}

Let
$$
A = \begin{bmatrix} 3 & 1 & 4 & 1 & 5 & 9 & 2 & 6\\ 2 & 7 & 1 & 8 & 2 & 8 &
1 & 8\\ 1 & 4 & 1 & 4 & 2 & 1 & 3 & 5 \end{bmatrix} , \vec{y} = \begin{bmatrix}
-7\\ 20\\ 8
\end{bmatrix}
$$


Find a 2-Space Vector \(\vec{x} \in \R^8\) so that \(\vec{y} = A \vec x\)

At least one of the following vectors are a solution:

$$
\vec{x_1} = \begin{bmatrix}
0\\ 0\\ 0\\ 0\\ a\\ b\\ 0\\ 0
\end{bmatrix}, \ \vec{x_2} = \begin{bmatrix}
a\\ b\\ 0\\ 0\\ 0\\ 0\\ 0\\ 0
\end{bmatrix}, \ \vec{x_3} = \begin{bmatrix}
0\\ 0\\ 0\\ a\\ b\\ 0\\ 0\\ 0
\end{bmatrix}, \ \vec{x_4} = \begin{bmatrix}
0\\ 0\\ 0\\ 0\\ 0\\ 0\\ a\\ b
\end{bmatrix}
$$

There are \({8 \choose 2} = 28\) 2-sparse vectors but only the four need to be considered.

\section{Solution}
\label{sec:org45e9626}

\subsection{Guiding Principles}
\label{sec:org2383883}
\(\vec y = A \vec x\) where \(\vec x\) contains free variables produces a system of
equations which can be organized into a matrix R. Row reduction is applied to
R to obtain discrete values \(a', \ b'\) for the free variables. \(a', \ b'\) are
then tested using the remaining equation. If the equation is valid, then \(x_i\) is a valid solution. Otherwise,
it is not.

\subsection{\(\mbox{\Large x_1}\)}
\label{sec:org49ea7f6}
\subsubsection{Organize \(\vec y = A x_1\) into a System of equations.}
\label{sec:org6d9aacf}
\begin{equation}
\begin{split}
\vec y = A x_1 & \\
\to & \begin{bmatrix} -7\\ 20\\ 8 \end{bmatrix} = \begin{bmatrix} 3 & 1 & 4 & 1 & 5 & 9 & 2 & 6\\ 2 & 7 & 1 & 8 & 2 & 8 &
1 & 8\\ 1 & 4 & 1 & 4 & 2 & 1 & 3 & 5 \end{bmatrix} \begin{bmatrix}
0\\ 0\\ 0\\ 0\\ a\\ b\\ 0\\ 0
\end{bmatrix}\\
\to & \begin{bmatrix}
-7\\ 20\\ 8
\end{bmatrix} = \begin{bmatrix}
5a + 9b\\
2a + 8b\\
2a + b
\end{bmatrix}\\
\to & \begin{bmatrix}
5 & 9 & -7\\
2 & 8 & 20\\
2 & 1 & 8
\end{bmatrix} \begin{bmatrix}
a\\ b\\ 1
\end{bmatrix}
\end{split}
\end{equation}

\subsubsection{Apply Row Reduction to solve free variables.}
\label{sec:orgabe1625}

\begin{equation}
\begin{split}
\begin{bmatrix}
5 & 9 & -7\\
2 & 8 & 20\\
2 & 1 & 8
\end{bmatrix} \underset{-r_3 + r_2}{\to}
\begin{bmatrix}
5 & 9 & -7\\
0 & 7 & 12\\
2 & 1 & 8
\end{bmatrix} \underset{-9r_3 + r_1}{\to}
\begin{bmatrix}
-13 & 0 & -79\\
0 & 7 & 12\\
2 & 1 & 8
\end{bmatrix}
\end{split}
\end{equation}


\begin{subequations}
\label{eq:Dustin}
\begin{equation}
-13a + 0 = & -79\\
\end{equation}
\begin{equation}
7b + 0 = & 12\\
\end{equation}
\begin{equation}
2a + b = & 8
\end{equation}
\end{subequations}

Solving for \(a',b'\) using (3a) and (3b) yields \(a' = \frac{79}{13}, \ b' =
\frac{12}{7}\)

\subsubsection{Plug values into the remaining equation}
\label{sec:org003354e}

Replacing \(a', \ b'\) in (3c) yields \(2(\frac{79}{13} + \frac{12}{7} \neq 8)\).
Thus \(\vec x_1\) is \textbf{not} a valid solution

\subsection{\(\mbox{\Large x_2}\)}
\label{sec:org1403c32}
\subsubsection{Organize \(\vec y = A x_2\) into a System of equations.}
\label{sec:orgc952f84}
\begin{equation}
\begin{split}
\vec y = A x_2 & \\
\to & \begin{bmatrix} -7\\ 20\\ 8 \end{bmatrix} = \begin{bmatrix} 3 & 1 & 4 & 1 & 5 & 9 & 2 & 6\\ 2 & 7 & 1 & 8 & 2 & 8 &
1 & 8\\ 1 & 4 & 1 & 4 & 2 & 1 & 3 & 5 \end{bmatrix} \begin{bmatrix}
a\\ b\\ 0\\ 0\\ 0\\ 0\\ 0\\ 0
\end{bmatrix}\\
\to & \begin{bmatrix}
-7\\ 20\\ 8
\end{bmatrix} = \begin{bmatrix}
3a + b\\
2a + 7b\\
a + 4b
\end{bmatrix}\\
\to & \begin{bmatrix}
3 & 1 & -7\\
2 & 7 & 20\\
1 & 4 & 8
\end{bmatrix} \begin{bmatrix}
a\\ b\\ 1
\end{bmatrix}
\end{split}
\end{equation}

\subsubsection{Apply Row Reduction to solve free variables.}
\label{sec:org0a01ff6}

\begin{equation}
\begin{split}
\begin{bmatrix}
3 & 1 & -7\\
2 & 7 & 20\\
1 & 4 & 8
\end{bmatrix} \underset{-2r_3 + r_2}{\to}
\begin{bmatrix}
3 & 1 & -7\\
0 & -1 & 4\\
1 & 4 & 8
\end{bmatrix} \underset{-4r_2 + r_3}{\to}
\begin{bmatrix}
3 & 1 & -7\\
0 & -1 & 4\\
1 & 0 & 24
\end{bmatrix}
\end{split}
\end{equation}


\begin{subequations}
\label{eq:Dustin}
\begin{equation}
3a + b = & -7\\
\end{equation}
\begin{equation}
0 + -b = & 4\\
\end{equation}
\begin{equation}
a + 0 = & 24
\end{equation}
\end{subequations}

Solving for \(a', \ b'\) using (6b) and (6c) yields \(a' = 24, \ b' = -4\)

\subsubsection{Plug values into the remaining equation}
\label{sec:orgdff08b7}

Replacing \(a', \ b'\) in (6a) yields \(3(24) - 4 \neq 7)\).
Thus \(\vec x_2\) is \textbf{not} a valid solution
\subsection{\(\mbox{\Large x_3}\)}
\label{sec:org92aea71}

\subsubsection{Organize \(\vec y = A x_3\) into a System of equations.}
\label{sec:org8cd83c7}
\begin{equation}
\begin{split}
\vec y = A x_3 & \\
\to & \begin{bmatrix} -7\\ 20\\ 8 \end{bmatrix} = \begin{bmatrix} 3 & 1 & 4 & 1 & 5 & 9 & 2 & 6\\ 2 & 7 & 1 & 8 & 2 & 8 &
1 & 8\\ 1 & 4 & 1 & 4 & 2 & 1 & 3 & 5 \end{bmatrix} \begin{bmatrix}
0\\ 0\\ 0\\ a\\ b\\ 0\\ 0\\ 0
\end{bmatrix}\\
\to & \begin{bmatrix}
-7\\ 20\\ 8
\end{bmatrix} = \begin{bmatrix}
a + 5b\\
8a + 2b\\
4a + 2b
\end{bmatrix}\\
\to & \begin{bmatrix}
1 & 5 & -7\\
8 & 2 & 20\\
4 & 2 & 8
\end{bmatrix} \begin{bmatrix}
a\\ b\\ 1
\end{bmatrix}
\end{split}
\end{equation}

\subsubsection{Apply Row Reduction to solve free variables.}
\label{sec:orgbac5438}

\begin{equation}
\begin{split}
\begin{bmatrix}
1 & 5 & -7\\
8 & 2 & 20\\
4 & 2 & 8
\end{bmatrix} \underset{-2r_3 + r_2}{\to}
\begin{bmatrix}
1 & 5 & -7\\
0 & -2 & 4\\
4 & 2 & 8
\end{bmatrix} \underset{r_2 + r_3}{\to}
\begin{bmatrix}
1 & 5 & -7\\
0 & -2 & 4\\
4 & 0 & 12
\end{bmatrix} \underset{\frac{1}{4}r_3, -\frac{1}{2} r_2}{\to}
\begin{bmatrix}
1 & 5 & -7\\
0 & 1 & -2\\
1 & 0 & 3
\end{bmatrix}
\end{split}
\end{equation}


\begin{subequations}
\label{eq:Dustin}
\begin{equation}
a + 5b = & -7\\
\end{equation}
\begin{equation}
0 + b = & -2\\
\end{equation}
\begin{equation}
a + 0 = & 3
\end{equation}
\end{subequations}

Solving for \(a', \ b'\) using (9b) and (9c) yields \(a' = 3, \ b' = -2\)

\subsubsection{Plug values into the remaining equation}
\label{sec:orgb3ba81b}

Replacing \(a', \ b'\) in (9a) yields \(3 + 5(-2) = -7\).
Thus \(\vec x_3\) \textbf{is} a valid solution
\subsection{\(\mbox{\Large x_4}\)}
\label{sec:orgf2f5b3e}

\subsubsection{Organize \(\vec y = A x_4\) into a System of equations.}
\label{sec:orgd835fc9}

\begin{equation}
\begin{split}
\vec y = A x_4 & \\
\to & \begin{bmatrix} -7\\ 20\\ 8 \end{bmatrix} = \begin{bmatrix} 3 & 1 & 4 & 1 & 5 & 9 & 2 & 6\\ 2 & 7 & 1 & 8 & 2 & 8 &
1 & 8\\ 1 & 4 & 1 & 4 & 2 & 1 & 3 & 5 \end{bmatrix} \begin{bmatrix}
0\\ 0\\ 0\\ 0\\ 0\\ 0\\ a\\ b
\end{bmatrix}\\
\to & \begin{bmatrix}
-7\\ 20\\ 8
\end{bmatrix} = \begin{bmatrix}
2a + 6b\\
a + 8b\\
3a + 5b
\end{bmatrix}\\
\to & \begin{bmatrix}
2 & 6 & -7\\
1 & 8 & 20\\
3 & 5 & 8
\end{bmatrix} \begin{bmatrix}
a\\ b\\ 1
\end{bmatrix}
\end{split}
\end{equation}

\subsubsection{Apply Row Reduction to solve free variables.}
\label{sec:org94902e4}

\begin{equation}
\begin{split}
\begin{bmatrix}
2 & 6 & -7\\
1 & 8 & 20\\
3 & 5 & 8
\end{bmatrix} \underset{-3r_2 + r_3}{\to}
\begin{bmatrix}
2 & 6 & -7\\
1 & 8 & 20\\
0 & -19 & -52
\end{bmatrix} \underset{- \frac{8}{19} r_3 + r_2}{\to}
\begin{bmatrix}
2 & 6 & -7\\
1 & 0 & \frac{160}{19}\\
0 & -19 & -52
\end{bmatrix}
\end{split}
\end{equation}


\begin{subequations}
\label{eq:Dustin}
\begin{equation}
2a + 6b = & -7\\
\end{equation}
\begin{equation}
a + 0 = & \frac{160}{19}\\
\end{equation}
\begin{equation}
0 - 19b = & -52
\end{equation}
\end{subequations}

Solving for \(a', \ b'\) using (12b) and (12c) yields \(a' = \frac{160}{19}, \ b' = \frac{52}{19}\)

\subsubsection{Plug values into the remaining equation}
\label{sec:org017b201}

Replacing \(a', \ b'\) in (12a) yields \(2(\frac{160}{19}) + 6(\frac{52}{19}) = -7\).
Thus \(\vec x_4\) is \textbf{not} a valid solution
\end{document}
